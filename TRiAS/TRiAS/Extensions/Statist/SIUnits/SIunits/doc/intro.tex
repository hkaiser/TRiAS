%  //-=-=-=-=-=-=-=-=-=-=-=-=-=-=-=-=-=-=-=-=-=-=-=-=-=-=-=-=-=-=-=-=-=-=-\\
%  // NOTICE:                                                             \\
%  //   This file is part of "The SI Library of Unit-Based Computation."  \\
%  //   See "doc/README" for copyright, license, and other information.   \\
%  //-=-=-=-=-=-=-=-=-=-=-=-=-=-=-=-=-=-=-=-=-=-=-=-=-=-=-=-=-=-=-=-=-=-=-\\


% <!------------------------------------------------------------>


\documentstyle[twoside,makeidx]{article}
\flushbottom
\pagestyle{headings}

\setlength{\topmargin}{-0.5in}
\setlength{\textwidth}{5.5in}
\setlength{\oddsidemargin}{.5in}
\setlength{\evensidemargin}{.5in}
\setlength{\textheight}{9in}

\addtolength{\parskip}{2pt}

\setcounter{tocdepth}{1}

\renewcommand{\thefootnote}{}

\def\eg{{\it e.g.}}
\def\etal{{\it et al.}}
\def\etc{{\it etc.}}
\def\ie{{\it i.e.}}
\def\sic{{\it sic}}

% cpp macro  produce "C++" as logo
\def \cpp{\leavevmode\hbox{C{\raise.15ex\hbox{+}}\kern-.05em{\raise.15ex\hbox{+}}}\ignorespaces}

% SIlib macro: produce "SI Library" in small caps
\def \SIlib{{\sc SI~Library}}

% lib macro: produce "Library" in small caps
\def \lib{{\sc Library}}


% <!------------------------------------------------------------>


\pagenumbering{roman}
\begin{document}

\title{ Introduction to\\the \SIlib\ of Unit-Based Computation
  \footnote{
  An expanded version of this paper is in preparation;
  please contact the author for details.}
}

\author{Walter E.\ Brown\thanks{Email: wb@fnal.gov}\\
  {\sc Fermi National Accelerator Laboratory}
}

  \footnotetext{
  {\it Copyright Notice}:
  This manuscript has been authored
  by Universities Research Association, Inc.,
  under contract No. DE-AC02-76CH03000
  with the U.S. Department of Energy.
  The United States Government retains
  and the publisher, by accepting the article for publication, acknowledges
  that the United States Government retains
  a nonexclusive, paid-up, irrevocable, worldwide license
  to publish or reproduce the published form of this manuscript,
  or allow others to do so,
  for United States Government purposes.

  {\it Credit line}:
  Work supported by the U.S. Department of Energy
  under contract No. DE-AC02-76CH03000.

  {\it Distribution}:
  Approved for public release;
  further dissemination unlimited. 

  {\it Disclaimer}:
  This report was prepared
  as an account of work
  sponsored by an agency of the United States Government.
  Neither the United
  States Government nor any agency thereof,
  nor any of their employees,
  makes any warranty, expressed or implied,
  or assumes any legal liability or responsibility
  for the accuracy, completeness, or usefulness
  of any information, apparatus, product, or process disclosed,
  or represents that its use would not infringe privately owned rights.
  Reference herein to any specific commercial product, process, or service
  by trade name, trademark, manufacturer, or otherwise,
  does not necessarily constitute or imply
  its endorsement, recommendation, or favoring
  by the United States Government or any agency thereof.
  The views and opinions of authors expressed herein
  do not necessarily state or reflect
  those of the United States Government or any agency thereof.
  }

\date{September 2, 1998}

\maketitle



% <!------------------------------------------------------------>


\begin{flushright}
{\it Ye shall do no unrighteousness in judgment,\\
     in measures of length, of weight, or of quantity.\\
     -- Leviticus 19:35}
\end{flushright}

%\begin{flushright}
%{\it 
%     And these shall be the measures thereof.\\
%     -- Ezekiel 48:16}
%\end{flushright}

\begin{quotation} \tableofcontents \end{quotation}

\newpage \setcounter{page}{1}\pagenumbering{arabic}


% <!------------------------------------------------------------>


\section{ Introduction }

\subsection{ Units-checking and type-safety }

Scientific work commonly deals with numbers
that represent amounts of physical dimensions.
Scientists are therefore trained,
as a matter of routine,
to be very careful with these numbers' units,
lest incompatible values be accidentally combined in calculation
or lest incorrect units be ascribed to the outcome.
As Halliday and Resnick exhort \cite[pp. 35-6]{Halliday}:
\begin{quote}
\begin{itemize}
  \item
    ``In carrying out any calculation,
    always remember to attach the proper units to the final result,
    for the result is meaningless without this....''
  \item
    ``One way to spot an erroneous equation
    is to check the dimensions of all its terms....''
  \item
    ``In any legitimate physical equation
    the dimensions of all the terms must be the same....''
\end{itemize}
\end{quote}

In modern digital computation,
the analogous concept is known as {\it static type-checking}.
This concept lies at the core
of such object-oriented programming languages as \cpp,
and yields among the most valuable benefits
of such software methodology.
To demonstrate the analogy's accuracy,
we paraphrase Halliday and Resnick above (differences {\bf emphasized}):
\begin{quote}
\begin{itemize}
  \item
    In {\bf programming} any calculation,
    always remember to attach the proper {\bf data type} to the final result,
    for the result is meaningless without this.
  \item
    One way to spot an erroneous {\bf program}
    is to check the {\bf data types} of all its {\bf objects}.
  \item
    In any legitimate {\bf assignment}
    the {\bf data types} of {\bf both} the {\bf sides} must be the same.
\end{itemize}
\end{quote}

A substantial body of theoretical and applied computing research
has been devoted to type-checking.
As summarized by Fagan \cite[pp. ii, 2]{Fagan}, static typing provides
    ``... important feedback to the programmer
    by detecting a large class of program errors before execution''
as well as
    ``... a succinct, intelligible form of program documentation,
    making programs easier to read and understand.''
Fagan further notes that
  ``simple conceptual errors''
often manifest as type errors,
that these are detectable by type-checking software
(such as a compiler)
used to ensure
  ``... the type documentation is consistent,''
and (citing \cite{Gannon}) that
  ``[e]xperimental results estimate the number of ... type faults
  as something between 30\% and 80\% of all program errors.''
For these and related reasons,
programmers have long been taught to employ
and rely on type-checking.

\subsection{ Prior work }

There have been several earlier efforts
to attain these benefits.
Perhaps the most useful to scientific programmers
is currently found in the {\tt Units} subdirectory
of the {\it CLHEP} library.
Two files \cite{Maire1, Maire2}
define symbols for units and important constants
in common use by the high-energy physics community.

The use of these and similar symbols
provides, of course, an important boost
to code readability and correctness.
However, {\it CLHEP}'s {\tt Units} files
do not express the concepts of the various intended dimensions
to which its constants and units belong.
All these symbols have type {\tt HepDouble},
typically a synonym
for \cpp's {\tt double}.
Only in a relativistic model
where the speed of light is one
should it be possible to add meters to seconds;
it is otherwise highly unlikely to be a meaningful expression,
and thus highly likely
to reflect a conceptual and attendant computational error.

\subsection{ Current programming practice }

Given modern programming languages'
significant expressive capabilities
to design data types
tailored to the problem domain at hand,
we clearly now have both ample motivation and ample technology
to embrace type-safe object-oriented techniques
in all contemporary applications of computer programming.
Alas, in computer programming as practiced today,
the standard of care recommended above
seems only rarely applied to numeric quantities.
Informal inspection
of contemporary code samples
revealed that,
in numeric programming,
programmers make heavy, near-exclusive, use
of a language's native numeric types (\eg, {\tt double}).
A search of representative on-line archives did not even find
any queries on the subject,
suggesting little or no interest in, or knowledge of, alternatives.

In light of prior software art, this finding is not surprising.
However, such overly-general practice is a common source of errors:
it fails to distinguish
the diverse intentions
(\eg, distances, masses, energies, momenta, \etc)
that any such purely numeric value can represent.
Were such intentions routinely expressed
in our computer programs,
application of contemporary compiler technology
would routinely provide us
the benefits of type-safety in our numeric computations
as a by-product of the translation process.
This benefit is consistent with the unit-based approach
so strongly advised by \cite{Halliday},
among many others.

\subsection{ Scope of current project }

To address current deplorable practices in numeric computation,
we set out to develop a software subsystem
to provide a convenient means
of expressing, computing with, and displaying
numeric values with attached units,
thus obtaining the well-known benefits of type safety
consistent with recommended unit-based practices of long standing.
An additional requirement of this project
was to ensure strict compile-time type-checking
without run-time overhead
(\ie, at no run-time cost in time or in space).

More specifically, we sought
\begin{quote}
\begin{enumerate}
  \item
    application of current software technology
    to numeric physical concepts,
  \item
    convenience of expression in such application,
  \item
    general utility rooted in existing standards,
  \item
    use of nomenclature from our problem domain, and
  \item
    no attendant performance penalties!
\end{enumerate}
\end{quote}
The present project, known as
{\it The \SIlib\ of Unit-based Computation},
has succeeded in addressing these requirements.
The resulting software module
(known hereinafter as the \SIlib\ 
or, simply, the \lib)
meets (and, in many respects, greatly exceeds!) all its goals
and is intended for contribution
(for non-commercial use)
to the FPCLTF (``Zoom'') project library at Fermilab.


% <!------------------------------------------------------------>


\section{ Underpinnings and Basic Features }

\subsection{ International Standard of Units }

{\it Le Syst\`{e}me international d'Unit\'{e}s} (SI) \cite{BIPM}
is the recognized international standard
for describing measurable quantities and their units.
SI specifies seven mutually independent base quantities (dimensions):
{\it length},
{\it mass},
{\it time},
{\it electric current},
{\it thermodynamic temperature},
{\it amount of substance}, and
{\it luminous intensity}.
The base units for describing amounts of these
are specified, respectively, as the
{\it meter},
{\it kilogram},
{\it second},
{\it ampere},
{\it kelvin},
{\it mole}, and
{\it candela}.

In addition, SI describes a consistent system
for expressing new dimensions
(\eg, energy) in terms of the seven base dimensions.
In particular,
it includes a list of 21 such derived dimensions
whose amounts are described
in specially-named composites (\eg, joule)
of the base units.
Finally,
SI provides a list of prefixes
for forming units' decimal multiples and sub-multiples.

\subsection{ Library basics }

At its core,
{\it The \SIlib\ of Unit-based Computation} provides,
in the form of data types,
all the base dimensions specified by SI.
Further, it provides the ability
to declare additional data types to represent derived dimensions,
also as specified.

For programming convenience,
a very significant number of derived dimensions
have been included in the \lib.
In particular,
virtually all the dimensions described
by Horvath \cite{Horvath}
and by Pennycuick \cite{Pennycuick}
are provided.
A programmer is free to use, ignore, or add to these
as may be convenient
for the application at hand.

Similarly, all the base and composite units specified by SI
are provided in the \lib,
as are forms of the SI-mandated prefixes.
An extensive collection of diverse units
from published sources
(\eg, \cite{Horvath,
	    Pennycuick,
	    ParticleBook,
	    Maire1,
	    Maire2})
have also been included in the module;
these encompass traditional MKS, CGS, UK, and US units.
Many units not in common use have also been provided
in order to demonstrate both the diversity and the flexibility
that the \lib\ enables.

A significant collection of constants of nature
is also incorporated in the \SIlib.
Because any such constant can be used as a unit
(\eg, Mach is based on the speed of sound),
these constants of nature have been internally combined
with traditional units data;
many of the same published sources
were used to furnish their values.
As before, a programmer may use, ignore, or extend this list.

The \SIlib's infrastructure
is furnished to a user program
in the form of header files
that make known
all the \lib-defined data types,
units, and constants of nature.
All these identifiers
are declared within a \cpp\ {\tt namespace}
so as to avoid introducing extraneous symbols.

% <!------------------------------------------------------------>


\section{ Annotated Example }

The following sample code first illustrates
the fundamentals of instantiating \SIlib\ objects
in connection with SI's basic {\tt Length} dimension.
Note that, in the absence of any explicit units,
the appropriate base unit (or combination of base units)
will be assumed.
Further note that any mismatch between units and
dimensions will give rise to a compile-time type error.

Part two illustrates computation and output
with {\tt Length} and {\tt Length}-related dimensions.
Note in particular that the units in which values
are displayed may be changed dynamically at will;
however, no such change affects the internal representation of the value.
While a change in the display of any of the seven SI base dimensions
may impact the subsequent display of derived dimensions,
no change to the display of a derived dimension
will affect the display of any other dimension.
\begin{quote}
\begin{verbatim}
#include "SIunits/stdModel.h"
#include <iostream>

int main()  {

  // Preparation:
  using namespace si;     // make most si:: symbols available
  using namespace si::abbreviations;
  using namespace std;         // make all std:: symbols available

  // -----  Part 1: illustrate instantiation/initialization  -----

  // Successful instantiations:
  Length d;               // initialized to 0 meters by default
  Length d2( 2.5 );       // 2.5 meters; same as 2.5 * m
  Length d3( 1.2 * cm );  // 1.2 centimeters; recorded as .012 * m
  Length d4( 5 * d3 );    // equivalent to 6.0 centimeters
  Length d5( 1.23 * pico_ * meter ); // 1.23e-12 * m
  Length d6( d2 + d3 );   // all dimensions match

  // Bad instantiation attempts; all detected at compile-time:
  Length d7( d2 * d3 );   // oops: an Area can't initialize a Length
  Length d8 = 3.5;        // oops: 3.5 is not a Length
  Length d9;              // so far so good, ...
    d9 = 3.5;             // ... oops: 3.5 is still not a Length

  // -----  Part 2: illustrate computation and output  -----

  // Display with default labels:
  cout << inch << endl;        // display "0.0254 m"

  // Prefer centimeter labels from now on:
  Length::showAs( cm, "cm" );  // set default display units
  cout << inch << endl;        // display "2.54 cm"

  // Compute/display:
  Length len( 2*cm );
  Width  wid( 3*cm );          // Width is a synonym for Length
  Area   a( len * wid );
  cout << a << endl;           // display "6 cm^2"

  // Prefer to label Area in square meters:
  Area::showAs( m*m, "m2" );   // set default display units
  cout << a << endl;           // display "0.0006 m2"
  cout << a*4 << endl;         // display "0.0024 m2"

  // But a volume reverts:
  cout << a * 4*cm << endl;    // display "24 cm^3"

  return 0;

}  // main()
\end{verbatim}
\end{quote}


% <!------------------------------------------------------------>


\section{ Additional Features }

\subsection{ Choice of data representation }

By default, the \lib\ 
employs the \cpp\ native {\tt double}
data type as its underlying representation.
Recognizing, however, that different programming projects
may require different underlying data representations (\eg,
  {\tt float},
  {\tt long double},
  {\tt complex<float>},
\etc),
\cpp's {\tt template} mechanism
was used throughout this \lib's implementation.
This allows a knowledgeable programmer
to specify the data representation desired.
The memory requirement
for each object of an \SIlib\ type
is exactly the same memory amount
that would be required
for an object of the underlying native type.

\subsection{ No run-time overhead }

Because all type-checking is handled at compile-time
and because of heavy use of inlining,
there is no run-time computational overhead beyond the
time taken for the necessary arithmetic.
Thus, use of the \SIlib\ incurs no performance penalty
relative to computation on the native type.

\subsection{ Choice of calibration }

Because different users work
with values of radically different magnitudes,
the \SIlib\ allows for certain user calibration.
By default,
each of the seven base units
(meter, kilogram, second, \etc)
is calibrated to a value of one.
However, for users who customarily work, say,
in microns and nanoseconds,
it is possible to calibrate the \lib\ 
to treat these units as the base units instead.
Such calibration, however, can only take place
at the time the \lib\ is built;
dynamic recalibration is neither currently possible
nor envisioned as a future enhancement of the \lib.

\subsection{ Choice of model }

Finally, different programming projects
may use different models of the universe.
For example, to simplify certain computations,
high-energy physics calculations
often assume the speed of light to be one.
Such assumptions are also possible
within the \SIlib\ 
and all consequences of such assumptions
(\eg, the merging of the {\tt Length} and {\tt Time} dimensions)
can be accurately modeled.

The \lib\ is supplied with five models;
these are known, respectively, as the
  {\it standard},
  {\it relativistic},
  {\it high-energy physics},
  {\it quantum},
  and {\it natural}
models.
The following table provides some details
as to each model's characteristics;
all models use the {\it mole} as the unit of {\tt AmountOfSubstance}
and the {\it candela} as the unit of {\tt LuminousIntensity}:
\begin{center}
\samepage
\begin{tabular}{l l l}
  \multicolumn{1}{c}{ {\bf Model} }
& \multicolumn{1}{l}{ {\bf Defining Characteristics} }
& \multicolumn{1}{l}{ {\bf Default Units} }
\\
\hline
\hline
stdModel
& per Syst\`eme international
& \parbox{2.5in}{m ({\tt Length}); kg ({\tt Mass}); s ({\tt Time});
  \newline A ({\tt Current}); K ({\tt Temperature})}
\\
\hline
relModel
& $c = 1$
& \parbox{2.5in}{s ({\tt Length}, {\tt Time}); eV ({\tt Mass});
  \newline A ({\tt Current}); K ({\tt Temperature})}
\\
\hline
hepModel
& $c = k = e^{+} = 1$
& \parbox{2.5in}{ns ({\tt Length}, {\tt Time});
  \newline GeV ({\tt Mass}, {\tt Temperature}); $e^{+}$ (Charge)}
\\
\hline
qtmModel
& $c = k = \hbar = 1$
& \parbox{2.5in}{inverse GeV ({\tt Length}, {\tt Time});
  \newline GeV ({\tt Mass}, {\tt Temperature}, {\tt Current})}
\\
\hline
natModel
& $c = k = \hbar = G = 1$
& \parbox{2.5in}{numeric only ({\tt Length},
  \newline {\tt Time}, {\tt Mass}, {\tt Temperature}, {\tt Current})}
\\
\hline
\end{tabular}
\end{center}

An application program selects the desired model
via an \verb:#include "...": directive.
A programmer pays no price for the availability of multiple models;
only the code associated with the chosen model
need be linked with application code.
Additional models may, of course,
be added to the \lib\ by a knowledgeable programmer.

The \lib's five models are designed to provide
progressively more restrictive viewpoints.
Such ``forward compatibility'' in models
is obtained via recompilation using a header
from any more restrictive model,
and no other source change.
Thus, a user may freely develop code under one model,
and later elect to recompile and run with any more restrictive model.


% <!------------------------------------------------------------>


\section{ Concluding Remarks }

This paper has presented the concept of type-checking
as the computer analog of manual calculation with units.
The benefits of type-checking have been set forth,
together with empirical evidence
that programmers do not today apply type-checking to its fullest potential
in numeric computation.

The thrust of the paper has been the general description
of {\it The \SIlib\ of Unit-based Computation},
a software subsystem
intended to apply the documented benefits of type-checking
to numeric computation
with no run-time overhead in either time or space.
Convenience of use, economy of application, and ease of extensibility
were primary objectives of the project.
Major features of the \SIlib\ 
have been discussed and illustrated.

We conclude that the \SIlib\ software project
has met each of its objectives, and has exceeded many of them.
To quote sample user reaction to date,
``[the \SIlib]
is the first really good reason I've seen
to switch from Fortran to \cpp.''


% <!------------------------------------------------------------>


\section{ Acknowledgements }

In addition to his general support
as chair of the Fermi Physics Class Library Task Force,
Mark Fischler
first suggested applying the core of the \SIlib\ 
to additional models.
He also defined the desired properties of the advanced physics models
and assisted materially
in validating the \lib's behavior
in these models.
Isi Dunietz
also assisted with some of this validation
and with proofing.


% <!------------------------------------------------------------>


\section{ Addendum }

Several weeks after this paper was prepared and delivered at CHEP'98,
a colleague pointed out the 1994 work of Barton and Nackman \cite{Barton}.
Preliminary analysis of their chapter ``Algebraic Structure Categories''
suggests some overlap with the present work,
which independently discovered some of the same techniques.
While Barton and Nackman term their code ``advanced examples,''
their efforts should not be overlooked.


% <!------------------------------------------------------------>


\begin{thebibliography}{99}
\addcontentsline{toc}{section}{References}

\bibitem{Barton}
  Barton, John J. and Lee R. Nackman.
    {\em Scientific and Engineering \cpp:
    An Introduction with Advanced Techniques and Examples}.
    Addison-Wesley,
    1994.
    ISBN 0-201-53393-6.

\bibitem{BIPM}
  Bureau International des Poids et Mesures.
  ``Le Syst\`{e}me international d'Unit\'{e}s.''
  S\`{e}vres Cedex, France,
  1998.

\bibitem{ParticleBook}
  Caso, Carlo, \etal
    ``Review of Particle Physics.''
    The European Physical Journal,
    C3 (1998) pp. 69-70, 73.

\bibitem{Fagan}
  Fagan, Mike.
    ``Soft Typing:  An Approach to Type Checking
      for Dynamically Typed Languages.''
    Technical Report 92-184,
    Rice University,
    March 31, 1992.

\bibitem{Gannon}
  Gannon, John D.
    ``An Experimental Evaluation of Data Type Conventions.''
    Communications of the ACM,
    20:8 (August, 1977),
    pp. 584-595.

\bibitem{Halliday}
  Halliday, David and Robert Resnick.
    {\em Fundamentals of Physics}.
    John Wiley \& Sons, Inc.,
    1970.
    SBN 471 34430 3.

\bibitem{Horvath}
  Horvath, Ari L.
    {\em Conversion Tables of Units in Science \& Engineering}.
    The Macmillan Press Ltd.,
    1986.
    ISBN 0-444-01150-1.

\bibitem{ISO}
  International Standards Organization, JTC1/SC22.
    {\em Programming Languages -- \cpp}.
    ISO/IEC FDIS 14882, 1998.

\bibitem{Maire1}
  Maire, Michel and Evgueni Tcherniaev.
    File:  {\tt Units/PhysicalConstants.h}.
    In {\it CLHEP} v1.3, CERN, 1998.

\bibitem{Maire2}
  Maire, Michel and Evgueni Tcherniaev.
    File:  {\tt Units/SystemOfUnits.h}.
    In {\it CLHEP} v1.3, CERN, 1998.

\bibitem{Pennycuick}
  Pennycuick, Colin J.
    {\em Conversion Factors:  S. I. Units and Many Others}.
    Univ. of Chicago Press,
    1988.
    ISBN 0-226-65507-5.

\end{thebibliography}


% <!------------------------------------------------------------>


\end{document}
